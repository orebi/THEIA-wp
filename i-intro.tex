\section*{Executive Summary}

The  development of new scintillators in combination with high-efficiency, fast photon detection opens up the possibility for discrimination of Cherenkov and scintillation signals in large-scale detectors, offering both high light yield and directional information even at low threshold.  Such a detector, situated in a fully-equipped deep underground laboratory, and utilizing developments in computing and reconstruction techniques, could achieve unprecedented levels of background rejection, thus enabling a physics program that would span topics in nuclear, high-energy, and astrophysics.  

The Theia experiment targets a broad physics program, including a next-generation neutrinoless double beta decay search capable of reaching into the normal hierarchy region of phase space, sensitivity to solar neutrinos, supernova neutrinos, nucleon decay searches, and measurement of the neutrino mass hierarchy and CP violating phase.  

This paper describes the technical breakthroughs that have led to the development of this detector concept, and the potential impact of such a detector on the fields of particle and astrophysics. 

\newpage

\tableofcontents
\setcounter{tocdepth}{5}
\newpage

\section{Introduction and THEIA Overview - GDOG, Bob, JRK, Michi}

The \textsc{Theia}  detector~\cite{asdc} leverages a tried and tested methodology in combination with novel, cutting-edge technology.  
The future of neutrino detection technology lies in massive, high-precision detectors, offering multiple channels for detection.  Current technology is constrained by the choice of target material: water detectors are limited in energy threshold and resolution by the overall light yield of the Cherenkov process, and scintillator detectors are limited in size by optical attenuation in the target itself, and in reconstruction of event direction by the isotropic nature of scintillation light.  

The newly-developed water-based liquid scintillator (WbLS)~\cite{wbls} offers a unique combination of  high light yield and low-threshold detection with attenuation close to that of pure water, particularly at wavelengths $>$ 400~nm.  
Use of this novel target material could allow separation of prompt, directional Cherenkov light from the more abundant, isotropic, delayed scintillation light.  This would be a huge leap forwards in neutrino detection technology, enabling the first low-threshold, directional neutrino detector.  Such a detector could achieve fantastic background rejection using directionality, event topology, and particle ID.  
WbLS chemistry also allows loading of metallic ions as an additional target for
particle detection, including: $^7$Li for charged-current solar neutrino detection; $^{\rm nat}$Gd
for neutron tagging enhancement; 
or isotopes that undergo double beta decay, facilitating a
neutrinoless double-beta decay (NLDBD) program.  The formula and principle of mass-produced WbLS have been developed and demonstrated at the Brookhaven National Laboratory Liquid Scintillator Development Facility.  Metal-doped samples have been produced with high stability, with loadings of up to several percent.  The instrumentation for large-scale liquid production is currently under design.

\textsc{Theia} would combine the use of a 30--100-kton WbLS target, doping with a number of potential isotopes, high efficiency and ultra-fast
timing photosensors, and a deep underground location.  A potential site is the Long Baseline Neutrino Facility (LBNF) far site, where \textsc{Theia} could operate in conjunction with the liquid argon tracking detector proposed by DUNE~\cite{dune}.  
The basic elements of this detector are being developed now in experiments such as WATCHMAN~\cite{wm}, ANNIE~\cite{annie} and SNO+~\cite{snopl}.  

A large-scale WbLS detector such as \textsc{Theia} can achieve an impressively broad program of physics topics, with enhanced sensitivity beyond that of previous detectors.  Much of the program hinges on the capability to separate prompt Cherenkov light from delayed scintillation.  This separation provides many key benefits, including:
\begin{itemize}
\item The ring-imaging capability of a pure water Cherenkov detector (WCD).  This enables a long-baseline program in a scintillation-based detector, with the additional benefit of low-threshold detection of hadronic events.

\item  Direction reconstruction using prompt Cherenkov photons.   This allows statistical identification of events such as solar neutrinos, which offer a rich physics program in their own right, as well as forming a background to many rare-event searches, including NLDBD and nucleon decay.

\item Low thresholds and good energy and vertex resolution using the abundant scintillation light.

\item Detection of sub-Cherenkov threshold scintillation light.  This provides excellent particle identification, including enhanced neutron tagging, detection of sub-Cherenkov threshold particles such as kaons in nucleon decay searches, and separation of atmospheric neutrino-induced neutral current backgrounds for inverse beta decay searches.

\end{itemize}

One of the most powerful aspects of \textsc{Theia} is the flexibility:  in the target medium itself, and even in the detector configuration.  The WbLS target can be tuned to meet the most critical physics goals at the time by modifying features of the target cocktail, including: the fraction of water vs scintillator; the choice of wavelength shifters and secondary fluors; and the choice of loaded isotope.  There is also the potential to construct a bag to contain isotope, and perhaps a higher scintillator-fraction target, in the centre of the detector, building on work by KamLAND-Zen~\cite{klz} and Borexino~\cite{bor}.  


\subsection{Detector configuration}
Include TWO baseline designs -- ideal and ``realistic'' (existing cavern)

Include description of phased deployment
e.g. refer to THEIA-i, THEIA-ii, THEIA-iii

Define the baseline design for each here, so that physics sections can simply refer back

e.g. THEIA-ii might be 50ktonne 10\% WbLS, 90\% coverage

whereas THEIA-iii might be the above with 5\% WbLS but a bag containing LS + Te/Xe
