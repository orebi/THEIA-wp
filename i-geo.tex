\subsection{Applied Antineutino Physics - S Dye for antinu group}
%\paragraph{Motivation}
%BRIEF intro to physics motivation, and status of the field: our major competitors
%\paragraph{XX with THEIA}
Electron antineutrinos stream freely from rapidly decaying fission products within nuclear reactors and from long-lived radioactive isotopes and their daughters within Earth \cite{agm15}. Important information about nuclear reactors, Earth, and the properties of neutrinos themselves comes from measuring the rate and energy spectrum of the interactions of these $1-10$ MeV antineutrinos. Detecting antineutrinos from nuclear reactors at short \cite{nucifer15,songs07} and long \cite{nudar13,snif10} distances monitors the operation and identifies the location and power of the reactor with applications for nuclear non-proliferation \cite{adam10}. Such detections also provide fundamental understanding of neutrinos \cite{reines53,reines76,jgl08}. Detecting antineutrinos from the nuclear cascades of thorium-232 and uranium-238 within Earth \cite{kl05} estimates terrestrial radiogenic heating \cite{gando13,agostini15}, leading to a more complete understanding of the composition, structure, and thermal evolution of our planet \cite{dye_etal15}. Global antineutrinos emerge from nuclear beta-minus decays, which produce a characteristic energy spectrum for each isotope. While the mixture of isotopes decaying within a source uniquely determines the energy spectrum of the emitted antineutrinos, neutrino oscillations distort the spectrum of detected antineutrinos in a pattern determined by the distance from the source. The rate and energy spectrum of global antineutrino interactions varies dramatically with surface location. The following discussion assumes Theia is located at SURF. Figure (surf_reac_geo.pdf) shows the energy spectrum of the predicted rate of antineutrinos at SURF.

Geo-neutrino observations probe the quantities and distributions of terrestrial heat-producing elements uranium and thorium. The quantities of these elements gauge global radiogenic power, offering insights into the origin and thermal history of the Earth. Spatial distributions reveal the initial partitioning and subsequent transport of these trace elements between metallic core, silicate mantle, and crust types. Ongoing observations at underground sites in Japan and Italy record the energies but not the directions of geo-neutrinos from uranium and thorium. Without directions pointing back to source regions, disentangling the signals from various reservoirs requires resolution of differing rates or energy spectra at separate sites. Due to limited statistics and perhaps insufficient site contrast, the observations at Japan and Italy do not yet measure distinct rates or energy spectra. Theia offers an opportunity to remedy this situation.
%What we bring to the table - pros of THEIA design \newline
%Sensitivity estimates with baseline design (one of THEIA i--iii)
%\paragraph{Detector Requirements}
%A summary of the impact of different detector choices i.e. what happens if we stray from the relevant baseline
