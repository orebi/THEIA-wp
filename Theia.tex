\documentclass[11pt,prd,letterpaper,amsmath,amssymb,final,nofootinbib
%linenumbers 
,unsortedaddress,superscriptaddress
]{revtex4-1}
% Fun and exciting LaTeX packages!

\pdfoutput=1

%Margins
\usepackage[top=0.9in, bottom=0.9in, left = 0.9in, right=0.9in]{geometry}

\usepackage{graphicx}% Include figure files 
%\usepackage{subfig}
%\usepackage{caption}
%\usepackage{subcaption}
\usepackage{dcolumn}% Align table columns on decimal point
\usepackage{bbm}% blackboard bold
\usepackage{textcomp}% I can't remember what this one does...
\usepackage{color}% Change font colors
\usepackage{gensymb}% Some nice math symbols...
\usepackage{wasysym}
\usepackage{amssymb}
\usepackage{enumerate}
\usepackage{geometry}
\usepackage{setspace}
%\usepackage{hyperref}
%\usepackage{lineno}
\setcounter{tocdepth}{3}

  
%\linenumbers
% Begin main document...
\begin{document}

\title{ THEIA: \\ {\small }}

\author{THEIA Collaboration}
\affiliation{Many and Varied}

\maketitle

\section*{Executive Summary}

\newpage

\tableofcontents
\setcounter{tocdepth}{5}
\newpage

\section{Introduction and THEIA Overview - GDOG, Bob, JRK, Michi}

Standard intro

Include description of phased deployment

e.g. refer to THEIA-i, THEIA-ii, THEIA-iii

Define the baseline design for each here, so that physics sections can simply refer back

e.g. THEIA-ii might be 50ktonne 10\% WbLS, 90\% coverage

whereas THEIA-iii might be the above with 5\% WbLS but a bag containing LS + Te/Xe

\section{Technology Developments}
\subsection{Water-based Liquid Scintillator - R. Svoboda}
\subsection{Photon Sensors - Matt W, others?}
\subsection{Reconstruction Techniques -- B. Wonsak \& M. Tsanov}

Including techniques and results (to date) for both low and high energy events

\section{Physics Sensitivities and Detector Requirements}\label{s:physics}



\subsection{High-Energy Physics}
\subsubsection{Detector Simulation}
Summary of simulation used in following sections, including e.g. high energy physics list, specific parameters for each detector configuration.
\subsubsection{Long Baseline - M Wilking for LBL group}
\paragraph{Motivation}
BRIEF intro to physics motivation, and status of the field: our major competitors
\paragraph{XX with THEIA}
What we bring to the table - pros of THEIA design \newline
Sensitivity estimates with baseline design (one of THEIA i--iii)
\paragraph{Detector Requirements}
A summary of the impact of different detector choices i.e. what happens if we stray from the relevant baseline
\subsubsection{Nucleon Decay - no one yet??}
\paragraph{Motivation}
BRIEF intro to physics motivation, and status of the field: our major competitors
\paragraph{XX with THEIA}
What we bring to the table - pros of THEIA design \newline
Sensitivity estimates with baseline design (one of THEIA i--iii)
\paragraph{Detector Requirements}
A summary of the impact of different detector choices i.e. what happens if we stray from the relevant baseline
\subsubsection{Atmospheric Neutrinos}
\paragraph{Motivation}
BRIEF intro to physics motivation, and status of the field: our major competitors
\paragraph{XX with THEIA}
What we bring to the table - pros of THEIA design \newline
Sensitivity estimates with baseline design (one of THEIA i--iii)
\paragraph{Detector Requirements}
A summary of the impact of different detector choices i.e. what happens if we stray from the relevant baseline

\subsection{Low-Energy Physics}
\subsubsection{Detector Simulation}
Summary of simulation used in following sections, including e.g. high energy physics list, specific parameters for each detector configuration.
\subsubsection{Neutrinoless Double Beta Decay - L Winslow, V Lozza for DBD group}
\paragraph{Motivation}
BRIEF intro to physics motivation, and status of the field: our major competitors
\paragraph{XX with THEIA}
What we bring to the table - pros of THEIA design \newline
Sensitivity estimates with baseline design (one of THEIA i--iii)
\paragraph{Detector Requirements}
A summary of the impact of different detector choices i.e. what happens if we stray from the relevant baseline
\subsubsection{Geoneutrinos - S Dye for antinu group}
\paragraph{Motivation}
BRIEF intro to physics motivation, and status of the field: our major competitors
\paragraph{XX with THEIA}
What we bring to the table - pros of THEIA design \newline
Sensitivity estimates with baseline design (one of THEIA i--iii)
\paragraph{Detector Requirements}
A summary of the impact of different detector choices i.e. what happens if we stray from the relevant baseline
\subsubsection{Sterile Neutrinos - no one yet??}
\paragraph{Motivation}
BRIEF intro to physics motivation, and status of the field: our major competitors
\paragraph{XX with THEIA}
What we bring to the table - pros of THEIA design \newline
Sensitivity estimates with baseline design (one of THEIA i--iii)
\paragraph{Detector Requirements}
A summary of the impact of different detector choices i.e. what happens if we stray from the relevant baseline


\subsection{Astrophysical Sources}
\subsubsection{Detector Simulation}
Summary of simulation used in following sections, including e.g. high energy physics list, specific parameters for each detector configuration.
\subsubsection{Solar Neutrinos - GDOG, R Bonventre}
\paragraph{Motivation}
BRIEF intro to physics motivation, and status of the field: our major competitors
\paragraph{XX with THEIA}
What we bring to the table - pros of THEIA design \newline
Sensitivity estimates with baseline design (one of THEIA i--iii)
\paragraph{Detector Requirements}
A summary of the impact of different detector choices i.e. what happens if we stray from the relevant baseline
\subsubsection{Supernova Neutrinos - no one yet??}
\paragraph{Motivation}
BRIEF intro to physics motivation, and status of the field: our major competitors
\paragraph{XX with THEIA}
What we bring to the table - pros of THEIA design \newline
Sensitivity estimates with baseline design (one of THEIA i--iii)
\paragraph{Detector Requirements}
A summary of the impact of different detector choices i.e. what happens if we stray from the relevant baseline
\subsubsection{Diffuse Supernova Neutrino Background - S Dye for antinu group}
\paragraph{Motivation}
BRIEF intro to physics motivation, and status of the field: our major competitors
\paragraph{XX with THEIA}
What we bring to the table - pros of THEIA design \newline
Sensitivity estimates with baseline design (one of THEIA i--iii)
\paragraph{Detector Requirements}
A summary of the impact of different detector choices i.e. what happens if we stray from the relevant baseline
\subsubsection{Indirect Dark Matter ?? }
From Michi -- there is still some interest in detecting decay/annihilation neutrinos in the multi-MeV range where THEIA would probably be able to place very competitive limits (as I noticed speaking to a theorist from Brussels a few weeks ago). It would be an addition to the antineutrino WG. Certainly, it?s rather low priority, but we could put it there as a place-holder.
\paragraph{Motivation}
BRIEF intro to physics motivation, and status of the field: our major competitors
\paragraph{XX with THEIA}
What we bring to the table - pros of THEIA design \newline
Sensitivity estimates with baseline design (one of THEIA i--iii)
\paragraph{Detector Requirements}
A summary of the impact of different detector choices i.e. what happens if we stray from the relevant baseline




%\section{Site Selection}
%
%Do we want to add this, or not really?  Maybe for the White Paper we should simply assume SURF?
%
%GDOG suggests to remove this section
%
%%\section{Baseline Detector Design}
%%\subsection{Detector Configuration Requirements}
%%Size, Depth, light yield, separation requirements
%%\subsection{WbLS Target}
%%\subsection{Photon Sensor Selection - Hybrid Scheme}
%
%\section{THEIA Timeline}
%
%Do we want to include anything on this?  Or does it simply risk that we are immediately out of date?
%
%GDOG suggests to remove this section


\section{Conclusions}

\begin{thebibliography}{00}

\bibitem{wbls}M. Yeh {\it et al.}, { ``A New Water-based Liquid Scintillator and Potential Applications.''} 
Nucl. Inst. \& Meth. A{\bf 660} 51 (2011).

\end{thebibliography}

\end{document}
