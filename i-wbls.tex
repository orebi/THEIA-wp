\subsection{Water-based Liquid Scintillator - R. Svoboda}


There are a number of demonstrations required in order to realize the conceptual detector presented here.  These include (but are not limited to): 
\begin{enumerate}
\item Sufficiently high intrinsic light yield and long attenuation length to meet minimal light collection requirements.   (This  requirement can be offset by high efficiency, high coverage photon detection).
\item Successful separation of  Cherenkov and scintillation signals, with sufficiently high Cherenkov light yield to maintain direction resolution and ring imaging capability.  This can be achieved by ultra-fast timing photon detection, such as LAPPDs, tuning of the WbLS cocktail, or a combination of the two.
\item Stability of the above properties over long timescales, and with respect to isotope loading ({\it e.g.} Gd, Li, Te).
\item Materials compatibility studies.
\item Demonstrated reconstruction \& particle ID capability.
\end{enumerate}

The R\&D program for \textsc{Theia} strongly leverages existing efforts and synergy with other programs, such as WATCHMAN~\cite{wm}, ANNIE~\cite{annie}, SNO+~\cite{snopl} and others.  Ongoing work includes WbLS development at BNL, purification and compatibility studies at UC Davis, characterization and optimization with the CHESS detector at UC Berkeley and LBNL~\cite{chess, chess2}, fast photon sensor development at ANL, U Chicago, Iowa State and others ~\cite{mcp--lappd3}, development of reconstruction algorithms~\cite{elagin, elagin2}, and potential nanoparticle loading in NuDot at MIT~\cite{nudot}.
